% This resume is a resume of Jinseop Kim's.
% This template, I have downloaded from "http://goo.gl/ir6bjP".

%%%%%%%%%%%%%%%%%%%%%%%%%%%%%%%%%%%%%%%%%
% Medium Length Professional CV
% LaTeX Template
% Version 2.0 (8/5/13)
%
% This template has been downloaded from:
% http://www.LaTeXTemplates.com
%
% Original author:
% Trey Hunner (http://www.treyhunner.com/)
%
% Important note:
% This template requires the resume.cls file to be in the same directory as the
% .tex file. The resume.cls file provides the resume style used for structuring the
% document.
%
%%%%%%%%%%%%%%%%%%%%%%%%%%%%%%%%%%%%%%%%%


%----------------------------------------------------------------------------------------
%	PACKAGES AND OTHER DOCUMENT CONFIGURATIONS
%----------------------------------------------------------------------------------------

\documentclass{resume} % Use the custom resume.cls style

\usepackage[left=0.75in,top=0.6in,right=0.75in,bottom=0.6in]{geometry} % Document margins
\usepackage{hyperref} % Hyerper link
% custom packages
\usepackage{amssymb}
\usepackage{kotex}

\name{김 진섭} % Your name
% \birthday{April 14, 1986}
% \profile-picture{./profile.jpg}
\address{경기도 xx시 xx동 ($\Box$$\Box$$\Box$-$\Box$$\Box$$\Box$)} % Your address
\address{010~$\cdot$~xxx~$\cdot$~xxxx \\ xxx@yahoo.com} % Your phone number and email

\begin{document}


%----------------------------------------------------------------------------------------
%	EDUCATION SECTION
%----------------------------------------------------------------------------------------

\begin{rSection}{학력}

{\bf 호서대학교 일반대학원} \hfill 2011.02~$\sim$~2013.02 \\
컴퓨터공학 석사 \\
학위논문:~\href{http://dlibrary.hoseo.ac.kr/search/searchDetail.do?rec_key=SH1_000000950591}
{\small ``스레드 생성 및 소멸 오버헤드를 줄이기 위한 히스토리기반 동적 스레드 풀''} \\
학점: 4.38/4.5

{\bf 호서대학교} \hfill 2004.02~$\sim$~2011.02 \\
컴퓨터공학 학사 \\
학점: 3.46/4.5

\end{rSection}


%----------------------------------------------------------------------------------------
%	WORK EXPERIENCE SECTION
%----------------------------------------------------------------------------------------

\begin{rSection}{경력}

  % \begin{rSubsection}{xxx}{Jan. 2016 - Present}{???}{Seoul, South Korea}
  % \item ???
  % \end{rSubsection}

  % ------------------------------------------------

  \begin{rSubsection}{SDC Micro, Inc.}{2012.12~$\sim$~현재}{연구원}{서울, 대한민국}
  \item 퀄컴 코리아 카메라 지그(공정) 업무 지원
    \vspace{1mm} \\
    \small{
      2013년 4월 부터 퀌컴 AP를 사용하는 삼성 스마트 폰의 카메라 센서 검증을 위한 소프트웨어 지원 \\
      주로 \textit{포팅} 및 \textit{기능검증}, \textit{디버깅} 업무 담당.}
  \end{rSubsection}

\end{rSection}
\vspace{-2mm}


%----------------------------------------------------------------------------------------
%	PERSONAL PROJECTS
%----------------------------------------------------------------------------------------

\begin{rSection}{개인 프로젝트}

  \begin{rSubsection}{HomeTV Service}{2015.09~$\sim$~현재}{}{}
  \item \small{
      % 동기:
      %  대게 부모님들이 그렇듯 저희 부모님도 스스로 토렌트 다운받아서 볼 정도로 인터넷을 잘하지 못 합니다.
      %  인터넷 TV를 사용하면 되지만, 대게 부모님들이 그렇듯 쓸데없이 돈나간 다며 인터넷 TV에 부정적 입니다.
      %  그래서 부모님이 즐겨보는 예능을 다운로드해 넣어드렸습니다.
      %  매일 토렌트를 검색하고 다운로드 받아 넣는 것이 귀찮아져서 웹앱으로 인터넷 TV와 비슷한 서비스를 개발하게 되었습니다.
      %  (안방 TV의 스틱 PC를 사용하시고 계셔서 프로그램 배포하는데 어려움이 있습니다.
      %   그런 이유로 프로그램 제어권이 서버에 있는 웹앱으로 만들기로 했습니다.)
      % 프로그램 설명:
      %  클로저로 작성되었고, 규모는 아직까지 600줄 정도 됩니다.
      %  *TODO*
      % 느낀점:
      %  *TODO*
      부모님을 위한 ``토렌트를 이용한 인터넷TV 서비스''.}
  \end{rSubsection}

  \begin{rSubsection}{카메라 지그 커맨드(\small{GUI.ver})}{2015.08~$\sim$~현재}{}{}
  \item \small{
      % 동기:
      %  퀄컴 카메라 지그 업무에 사용하기 위한 간단한 CUI 프로그램은 있었습니다.
      %  사용하면서 아래와 같이 불편해서 재개발하게 되었습니다.
      %  - 이미지, 바이너리 파일을 보기위해 별도의 프로그램이 필요하기 때문에 간단한 테스트도 복잡하게 느낌.
      %  - 제어는 PC에서 액션은 핸드폰에 일어나다 보니 두리번 거리거나 한 손으로 핸드폰을 들고 있어야 해서 불편함.
      % 프로그램 설명:
      %  클로저로 작성되었고, 규모는 아직까지 1500줄 정도 됩니다.
      %  *TODO*
      % 느낀점:
      %  GUI 프로그래밍 경험이 거의 없어서 *정말* 어떻게 구현해야 하는지 감도 안잡히고,
      %  게다가 FP도 거의 처음이라 상당히 힘들었습니다. 또 GUI 테스트는 아직도 어떻게 작성해야 하는지 모르겠습니다.
      %  MVC를 적용하여 로직 부분만 깔끔하게 분리해 테스트하고 싶었는데, 아직까지 끈끈히 붙어있게 맘이 걸립니다.
      퀄컴에서의 ``카메라 지그'' 업무 중에 가볍게 사용하려고 만든 GUI 프로그램.}
  \end{rSubsection}

  \begin{rSubsection}{카메라 지그 자동화 테스트}{2014.02~$\sim$~2015.04}{}{}
  \item \small{
      % 동기:
      %  (모덺 x 버전 x 디버깅 횟수) 마다 테스트해야 했습니다.
      %  반복되는 일을 매번 꼼꼼히 한다는 것은 사실 상당히 어렵습니다.
      %  어느날 A 기능을 수정했는데, 전혀 상관없이 보이던 B 기능이 망가졌습니다.
      %  그걸 모르고 A 기능만 잘 되는지 확인하고 장비업체에게 바이너리를 보내줬다가 불평을 들은 적도 있습니다.
      %  그래서 이대로는 안될 것 같아, 일하면서 틈틈히 테스트를 자동화할 프로그램을 작성하였습니다.
      % 프로그램 소개:
      %  파이썬으로 작성되었고, 약 1년 정도 개발했습니다.
      %  규모는 테스트 포함 7천줄 정도 됩니다.
      %  `파이썬'의 유닛테스트 프레임워크와 유사하게 `카메라 지그' 테스트 프레임워크를 만들고,
      %  `파이썬'으로 테스트를 작성할 수 있도록 만들었습니다.
      % 느낀점:
      %  `파이썬'을 사용한 최초 프로젝트여서 `파이썬'이 이렇게 자유도가 높은 언어인 줄 몰랐습니다.
      %  처음 격어본 동적 타이핑은 상당히 편리하기는 했지만, 대신 유닛 테스트를 꼼꼼히 해야 했습니다.
      %  또 처음으로 TDD 방식으로 개발했는데, 리듬감 있는 템포로 개발 하다보니
      %  -주말이나 일 끝나고 새벽에 개발할 정도로- 재미있게 개발했던 것 같습니다.
      퀄컴에서의 ``카메라 지그'' 업무를 보면서 모델마다 반복되는 기능정검을 대신 하기위한 테스트 프로그램.}
  \end{rSubsection}

\end{rSection}
\vspace{-2mm}


%----------------------------------------------------------------------------------------
%	TECHNICAL STRENGTHS SECTION
%----------------------------------------------------------------------------------------

\begin{rSection}{기술 강점}

\begin{tabular}{ @{} >{\bfseries}l @{\hspace{2ex}} l }
  프로그래밍 언어 & C/C++, Clojure(Script), Python, Shell script \\
  운영체제 & Linux/Unix 시스템 프로그래밍, Linux 디바이스 드라이버 \\
  툴 & GIT, Emacs
\end{tabular}

\end{rSection}


\pagebreak

%----------------------------------------------------------------------------------------
%	SELF INTRODUCTION
% ----------------------------------------------------------------------------------------

\begin{rSection}{자기 소개}

TODO

\end{rSection}

\end{document}
