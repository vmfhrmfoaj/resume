% Document settings
% =================
\documentclass[line,margin]{res}

% Packages
% --------
\usepackage{graphicx}
\usepackage{hyperref}
\usepackage{kotex}
\usepackage{color}

% Set up
% ------
\hypersetup{colorlinks=false}
\renewcommand{\today}{\number\year.\ifnum\number\month<10 0\fi \number\month}

% Document
% ------->
\begin{document}


% Header
% ======
\name{김진섭{\small (1986.02.14)}}
\address{E-mail: vmfhrmfoaj@yahoo.com}


% Resume
% ======
\begin{resume}

  % Self-introduction
  % -----------------
  \section{자기소개}

  안녕하세요.
  %
  % 왜, 프로그래밍을 시작했지?
  % ~~~~~~~~~~~~~~~~~~~~~~
  제가 프로그래밍을 왜 공부하기 시작하였는지 정확히 기억나지 않지만, 영화에 나오는 해커들을 동경했었던 것 같습니다.
  그래서인지 해커를 연상시키는 유닉스\&리눅스 운영체제와 이맥스 에디터, Lisp 계열 언어들을 즐겨 사용합니다.


  % Hobbies
  % -------
  \section{취미}

  이맥스 설정 파일 관리, 일본 애니메이션 감상.


  % Skills
  % ------
  \section{스킬}

  {\sl Languages \& Software:} ~Lisp({\small Clojure, Emacs Lisp}), Python, C/C++ \\
  {\sl Operating systems:} ~Linux\&Unix


  % Portfolio
  % ---------
  \section{개인 프로젝트}

  {\sl 퀄컴 카메라 지그 기능 테스트 자동화 \hfill 2014.04 $\sim$ \today}
  \vspace{1mm}
  \newline
  {
    \small
    퀄컴 카메라 지그 업무 중에 장비업체에게 제공하는 기능들이 제대로 동작하는지 확인하는 업무도 있었습니다.
    반복되는 테스트를 대신하는 프로그램을 만들어 일부 테스트를 자동화 시켰습니다.
  }


  % Experience
  % ----------
  \section{경력사항}

  {\sl SDC Micro: 퀄컴 카메라 지그 업무 지원 \hfill 2013.04 $\sim$ \today}
  \vspace{1mm}
  \newline
  {
    \small
    퀄컴 칩을 사용하는 삼성 스마트 폰 모델의 카메라 모듈센서 검증을 위한 소프트웨어 지원,
    주로 ``포팅'' 및 ``기능 검증'' 업무를 담당하였습니다.
  }

  {\sl SDC Micro 입사 \hfill 2012.12}
  \vspace{0mm}


  % Education
  % ---------
  \section{학력사항}

  {\sl 호서대 일반대학원, 컴퓨터공학 석사 \hfill 2011.02 $\sim$ 2013.02}
  \vspace{-4mm}
  {
    \small
    \begin{itemize}
    \item[-] 학위논문: ~\href{http://dlibrary.hoseo.ac.kr/search/DetailView.ax?sid=4&cid=950591}
      {``스레드 생성 및 소멸 오버헤드를 줄이기 위한 히스토리기반 동적 스레드 풀''}
      \vspace{-1mm}
    \item[-] 학점: ~4.38/4.5
    \end{itemize}
  }

  {\sl 호서대학교, 컴퓨터공학과 \hfill 2004.02 $\sim$ 2011.02}
  \vspace{-4mm}
  {
    \small
    \begin{itemize}
    \item[-] 학점: ~3.46/4.5
    \end{itemize}
  }


  % Open source
  % -----------
  \section{오픈소스 활동}

  {\sl GitHub:} \href{http://github.com/vmfhrmfoaj/}{http://github.com/vmfhrmfoaj}

  {\sl 패치}
  \vspace{-4mm}
  {
    \small
    \begin{itemize}
    \item[-] Emacs({\footnotesize clojure-mode package}),
      \href{https://github.com/clojure-emacs/clojure-mode/pull/263}
      {\textcolor[gray]{0.5}{[rejected]} Fixed that wrong indentation}.
    \item[-] Emacs({\footnotesize clojure-mode package}),
      \href{https://github.com/clojure-emacs/clojure-mode/pull/257}
      {Fix for that wrong indentation in multi-arity forms}.
    \item[-] Emacs({\footnotesize aggressive-indent-mode package}),
      \href{https://github.com/Bruce-Connor/aggressive-indent-mode/pull/30}
      {Prevent that to replace the 'message' func to 'ignore' func}.
    \end{itemize}
  }

  {\sl 버그 리포팅}
  \vspace{-4mm}
  {
    \small
    \begin{itemize}
    \item[-] Emacs({\footnotesize ido-vertical-mode package}),
      \href{https://github.com/gempesaw/ido-vertical-mode.el/issues/14#issuecomment-74077234}
      {Find-file that doesn't exist doesn't display confirmation dialog}.
    \item[-] Emacs,
      \href{https://lists.gnu.org/archive/html/bug-gnu-emacs/2014-11/msg00518.html}
      {bug\#5662: flet not undone on lisp nesting error}.
    \end{itemize}
  }

\end{resume}

% Document
% <-------
\end{document}
