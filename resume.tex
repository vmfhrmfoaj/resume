% Document settings
% =================
\documentclass[line,margin]{res}

% Packages
% --------
\usepackage{graphicx}
\usepackage{hyperref}
\usepackage{kotex}

% Set up
% ------
\hypersetup{colorlinks=false}

% Document
% ------->
\begin{document}


% Header
% ======
\name{김진섭{\small (1986.02.14)}}
\address{H.P: 010-8633-9775}
\address{E-mail: vmfhrmfoaj@yahoo.com}


% Resume
% ======
\begin{resume}

  % Self-introduction
  % -----------------
  \section{자기소개}

  % XXX: 프로그램으로써의 나는?

  안녕하세요.
  저는 고등학교 때 C를 접하면서 프로그래밍을 시작해 x86어셈블리와 C++, 파이썬, 리스프를 익혔으며,
  최근에 주로 사용하는 언어는 파이썬과 리스프 입니다. \\
  x86어셈블리는 C언어 디비겅 하면서 우연히 접하였고 독학으로 배웠습니다.
  최적화 한다고 과제로 작성한 C 프로그램을 어셈블리로 재작성하고 하였는데, 지금 생각해보면 바보같은 행동이였습니다.
  프로세서 성능이 많이 좋아져 명령어 몇 개 아끼거나 레지스터를 효율적으로 사용한다거나 하는 방식으로는 큰 차이를 볼수 없고,
  성능에 영향을 미치는 부분은 몇 안되서 프로파일러로 핫 스폿들을 찾아 해당 부분만 어셈블리로 작성하는 방식으로도 비슷하 결과를 얻을 수 있는데 말입니다. \\
  C++는 OOP를 공부하면서, 파이썬은 인터프리터 언어 하나는 사용할 수 있어야 편할 것 같아 익혔고,
  리스프는 이맥스 기능을 확장하거나 동작을 변경할 때 사용하면서 저절로 익혀졌습니다.

  대학생 시절 리눅스 수업을 듣고 커널 공부하면서 간단한 디바이스 드라이버를 작성하였고,
  파견근무 때도 커널 이슈는 관심을 가지고 분석 및 해결 하였습니다.
  수업을 들은 이후 유닉스\&리눅스만 6년째 사용하고 있습니다.
  가장 좋아하는 리눅스 배포판은 젠투 리눅스와 리눅스 민트입니다.

  일하면서는 반복되는 업무를 자동화 시키기 위해 mmm-bat\footnote{
    \href{{http://github.com/vmfhrmfoaj/mmm-bat/tree/develop}}
    {http://github.com/vmfhrmfoaj/mmm-bat/tree/develop}}과
  jig-auto-test\footnote{
    \href{{http://github.com/vmfhrmfoaj/jig-auto-test/tree/public}}
    {http://github.com/vmfhrmfoaj/jig-auto-test/tree/public}}
  를 개발하였습니다. \\
  mmm-bat은 Bash를 사용해 안드로이드 부분 빌드시 컴파일 되는 오브젝트 파일들을
  ADB를 사용해 핸드폰으로 넣어주는 스크립트를 생성하는 프로그램입니다. \\
  jig-auto-test는 파이썬을 사용해 카메라 지그에서 제공하는 명령이 정상적으로 동작하는지 확인하는 자동화된 회귀 테스트입니다.
  카메라 지그에 새로운 기능을 추가할 때 TDD 방식으로 개발하기 위해 파이썬의 유닛 테스트를 흉내내서 만들었습니다. \\
  그리고 주말이나 시간이 남으면 융용한 이맥스 플러그인\footnote{
    \href{http://github.com/search?q=user\%3Avmfhrmfoaj\&type=Repositories\&ref=advsearch\&l=Emacs+Lisp}
    {http://github.com/search?q=user\%3Avmfhrmfoaj\&type=Repositories\&ref=advsearch\&l=Emacs+Lisp}}을
  개발하고 있습니다.

  최근에는 코드의 품질과 코드 재사용, 리펙토링, 디자인패턴과 같은 것에 관심을 가지게 되었고,
  자연스럽게 BDD나 TDD와 같은 개발방법론에 관심을 가지고 공부하고 있습니다.
  덩달아 클로저{\footnotesize {\sl Clojure}}와 같은 자바 기반의 함수형 언어도 공부하고 있습니다.


  % Skills
  % ------
  \section{스킬}

  {\sl Languages \& Software:} ~C/C++, Python, Lisp. \\
  {\sl Operating systems:} ~Linux.


  % Experience
  % ----------
  \section{경력사항}

  {\sl 퀄컴 카메라 지그 업무 지원} \hfill {\sl 2013.04 $\sim$ 2014.07} \vspace{1mm} \\
  {\small
    퀄컴 칩을 사용하는 삼성 핸드폰 모델에 대한 카메라 모듈센서 검증을 위한 소프트웨어 지원, \\
    주로 아래와 같은 업무를 담당함.}
  \vspace{1mm}
  \begin{itemize}
    {\small
    \item[-] 각 모델 포팅 및 유지보수. \vspace{-1mm}
    \item[-] 기능검증.}
  \end{itemize}

  {\sl SDC Micro 입사} \hfill {\sl 2012.12 $\sim$ 2014.07} \vspace{0mm}


  % Education
  % ---------
  \section{학력사항}

  {\sl 호서대 일반대학원, 컴퓨터공학 석사} \hfill {\sl 2011.02 $\sim$ 2013.02} \vspace{1mm}
  \begin{itemize}
    {\small
    \item[-] 학점: ~4.38/4.5 \vspace{-1mm}
    \item[-] 학위논문: ~\href{http://dlibrary.hoseo.ac.kr/search/DetailView.ax?sid=4&cid=950591}
                     ``스레드 생성 및 소멸 오버헤드를 줄이기 위한 히스토리기반 동적 스레드 풀''}
  \end{itemize}

  {\sl 호서대학교, 컴퓨터공학과} \hfill {\sl 2004.02 $\sim$ 2011.02} \vspace{1mm}
  \begin{itemize}
    {\small
    \item[-] 학점: ~3.46/4.5}
  \end{itemize}


\end{resume}

% Document
% <-------
\end{document}
