% Document settings
% =================
\documentclass[line,margin]{res}

% Packages
% --------
\usepackage{graphicx}
\usepackage{hyperref}
\usepackage{kotex}

% Set up
% ------
\hypersetup{colorlinks=false}
\renewcommand{\today}{\number\year.\ifnum\number\month<10 0\fi \number\month}

% Document
% ------->
\begin{document}


% Header
% ======
\name{김진섭{\small (1986.02.14)}}
\address{E-mail: vmfhrmfoaj@yahoo.com}


% Resume
% ======
\begin{resume}

  % Self-introduction
  % -----------------
  \section{자기소개}

  안녕하세요.
  % FIXME: 입문해서 뭐? 어쩌라고!
  저는 고등학교 시절 C 언어를 시작으로 프로그래밍에 입문하였습니다.
  %
  % 왜, 프로그래밍을 시작했지?
  % ~~~~~~~~~~~~~~~~~~~~~
  프로그래밍을 왜 시작하였는지 정확히 기억나지 않지만, 영화에 나오는 해커들의 모습을 동경했었던 것 같습니다.
  그래서인지 해커를 연상시키는 유닉스\&리눅스 운영체제와 이맥스 에디터, Lisp 계열 언어를 즐겨 사용합니다.

  % 무엇을 배웠지?
  % ~~~~~~~~~~~
  대학교 시절에는 주로 마이크로프로세서와 리눅스 커널/시스템 프로그래밍을 주로 공부하였습니다.
  %
  기억에 남는 수업으로는 대학교 3학년 수업으로 임베디드 웹서버를 주제로 프로젝트를 진행하는 수업 이였습니다.
  리눅스 환경에서 C 언어로 HTTP의 GET 요청을 파싱해 정적 페이지를 보여주거나 간단한 CGI를 사용하는 프로젝트였습니다.
  %
  프로젝트를 진행하면서 가장 기억에 남는 것은 코드 수정한 후 알수 없는 오류를 해결하지 못 해
  처음부터 다시 작성하는 일이 였습니다. 이런 경험 때문인지 GIT과 같은 형상관리 도구에 관심이 많습니다. \\
  % 기억에 남는 것은 ``데드라인이 끝날 때쯤에는 새로운 기능 추가하지 말자, 백업하자!'' 정도.
  %
  대학원 시절에는 주로 프로그래밍 언어, 코드 품질과 재사용에 대해 관심을 가졌고
  리팩토링 방법과 BDD/TDD를 익히기 위해 노력하였습니다. \\
  %
  최근에는 함수형 언어에 관심을 가지고 공부하고 있습니다.

  % TODO: 무엇을 하고 싶지?
  % ~~~~~~~~~~~~~~~~~~~~


  % Skills
  % ------
  \section{스킬}

  {\sl Languages \& Software:} ~Lisp({\small Clojure, Emacs Lisp}), Python, C/C++, GIT \\
  {\sl Operating systems:} ~Linux\&Unix


  % Portfolio
  % ---------
  \section{개인 프로젝트}

  {\sl 퀄컴 카메라 지그 기능 테스트 자동화 \hfill 2014.04 $\sim$ \today}
  \vspace{1mm}
  \newline
  {
    \small
    퀄컴 카메라 지그 업무를 보면서 다양한 업무 중에 장비업체에게 제공하는 기능들이 제대로 동작하는지 확인하는
    업무도 있었습니다. 반복되는 테스트를 대신하는 프로그램을 만들어 대부분의 테스트를 자동화 시켰습니다.
  }


  % Experience
  % ----------
  \section{경력사항}

  {\sl SDC Micro: 퀄컴 카메라 지그 업무 지원 \hfill 2013.04 $\sim$ \today}
  \vspace{1mm}
  \newline
  {
    \small
    퀄컴 칩을 사용하는 삼성 스마트 폰 모델의 카메라 모듈센서 검증을 위한 소프트웨어 지원,
    주로 ``모델 포팅'' 및 ``기능 검증'' 업무를 담당하였습니다.
  }

  {\sl SDC Micro 입사 \hfill 2012.12 $\sim$ \today}
  \vspace{0mm}


  % Education
  % ---------
  \section{학력사항}

  {\sl 호서대 일반대학원, 컴퓨터공학 석사 \hfill 2011.02 $\sim$ 2013.02}
  \vspace{-4mm}
  {
    \small
    \begin{itemize}
    \item[-] 학점: ~4.38/4.5 \vspace{-1mm}
    \item[-] 학위논문: ~\href{http://dlibrary.hoseo.ac.kr/search/DetailView.ax?sid=4&cid=950591}
      {``스레드 생성 및 소멸 오버헤드를 줄이기 위한 히스토리기반 동적 스레드 풀''}
    \end{itemize}
  }

  {\sl 호서대학교, 컴퓨터공학과 \hfill 2004.02 $\sim$ 2011.02}
  \vspace{-4mm}
  {
    \small
    \begin{itemize}
    \item[-] 학점: ~3.46/4.5
    \end{itemize}
  }


\end{resume}

% Document
% <-------
\end{document}
