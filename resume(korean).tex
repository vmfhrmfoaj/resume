% This resume is a resume of Jinseop Kim's.
% This template, I have downloaded from "http://goo.gl/ir6bjP".

%%%%%%%%%%%%%%%%%%%%%%%%%%%%%%%%%%%%%%%%%
% Medium Length Professional CV
% LaTeX Template
% Version 2.0 (8/5/13)
%
% This template has been downloaded from:
% http://www.LaTeXTemplates.com
%
% Original author:
% Trey Hunner (http://www.treyhunner.com/)
%
% Important note:
% This template requires the resume.cls file to be in the same directory as the
% .tex file. The resume.cls file provides the resume style used for structuring the
% document.
%
%%%%%%%%%%%%%%%%%%%%%%%%%%%%%%%%%%%%%%%%%


%----------------------------------------------------------------------------------------
%	PACKAGES AND OTHER DOCUMENT CONFIGURATIONS
%----------------------------------------------------------------------------------------

\documentclass{resume} % Use the custom resume.cls style

\usepackage[left=0.75in,top=0.6in,right=0.75in,bottom=0.6in]{geometry} % Document margins
\usepackage{hyperref} % Hyerper link
% custom packages
\usepackage{amssymb}
\usepackage{kotex}

\name{김 진섭} % Your name
% \birthday{April 14, 1986}
% \profile-picture{./profile.jpg}
\address{경기도 xx시 xx동 ($\Box$$\Box$$\Box$-$\Box$$\Box$$\Box$)} % Your address
\address{010~$\cdot$~xxx~$\cdot$~xxxx \\ xxx@yahoo.com} % Your phone number and email

\begin{document}


%----------------------------------------------------------------------------------------
%	EDUCATION SECTION
%----------------------------------------------------------------------------------------

\begin{rSection}{학력}

{\bf 호서대학교 일반대학원} \hfill 2011.02~$\sim$~2013.02 \\
컴퓨터공학 석사 \\
학위논문:~\href{http://dlibrary.hoseo.ac.kr/search/searchDetail.do?rec_key=SH1_000000950591}
{\small ``스레드 생성 및 소멸 오버헤드를 줄이기 위한 히스토리기반 동적 스레드 풀''} \\
학점: 4.38/4.5

{\bf 호서대학교} \hfill 2004.02~$\sim$~2011.02 \\
컴퓨터공학 학사 \\
학점: 3.46/4.5

\end{rSection}


%----------------------------------------------------------------------------------------
%	WORK EXPERIENCE SECTION
%----------------------------------------------------------------------------------------

\begin{rSection}{경력}

  % \begin{rSubsection}{xxx}{Jan. 2016 - Present}{???}{Seoul, South Korea}
  % \item ???
  % \end{rSubsection}

  % ------------------------------------------------

  \begin{rSubsection}{SDC Micro, Inc.}{2012.12~$\sim$~현재}{연구원}{서울, 대한민국}
  \item 퀄컴 코리아 카메라 지그(공정) 업무 지원. \\
    \small{
      퀄컴 칩을 사용하는 삼성 스마트 폰의 카메라 모듈센서 검증을 위한 소프트웨어 지원, \\
      주로 \textit{포팅} 및 \textit{기능검증}, \textit{디버깅} 업무를 담당하였습니다.}
  \end{rSubsection}

\end{rSection}
\vspace{-2mm}


%----------------------------------------------------------------------------------------
%	PERSONAL PROJECTS
%----------------------------------------------------------------------------------------

\begin{rSection}{개인 프로젝트}

  \begin{rSubsection}{카메라 지그 자동화 테스트}{2014.02~$\sim$~현재}{}{}
  \item \small{
      ``퀄컴 카메라 지그'' 업무 중에 장비업체에게 제공하는 기능들을 정검하는 업무가 있었습니다. \\
      반복되는 기능정검을 대신 하기위한 테스트 프로그램을 작성하여 일부 기능정검을 자동화 시켰습니다. \\
      기능정검을 위해서는 기구의 도움을 받아야 하는 테스트들도 있기 때문에 모든 기능을 자동화 시키지 못 했습니다.}
  \end{rSubsection}

\end{rSection}
\vspace{-2mm}


%----------------------------------------------------------------------------------------
%	TECHNICAL STRENGTHS SECTION
%----------------------------------------------------------------------------------------

\begin{rSection}{기술 강점}

\begin{tabular}{ @{} >{\bfseries}l @{\hspace{2ex}} l }
  프로그래밍 언어 & C/C++, Clojure, Python, Shell script \\
  운영체제 & Linux/Unix 시스템 프로그래밍, Linux 디바이스 드라이버 \\
  툴 & GIT, Emacs
\end{tabular}

\end{rSection}

\end{document}
