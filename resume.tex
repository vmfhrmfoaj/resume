% Document settings
% =================
\documentclass[line,margin]{res}

% Packages
% --------
\usepackage{graphicx}
\usepackage{hyperref}
\usepackage{kotex}

% Set up
% ------
\hypersetup{colorlinks=false}

% Document
% ------->
\begin{document}


% Header
% ======
\name{김진섭{\small (1986.02.14)}}
\address{H.P: 010-8633-9775}
\address{E-mail: vmfhrmfoaj@yahoo.com}


% Resume
% ======
\begin{resume}

  % Self-introduction
  % -----------------
  \section{자기소개}

  안녕하세요.
  % FIXME: 입문해서 뭐? 어쩌라고!
  저는 고등학교 시절 C를 시작으로 프로그래밍에 입문하였습니다.
  %
  % 왜, 프로그래밍을 시작했지?
  % ~~~~~~~~~~~~~~~~~~~~~
  프로그래밍을 왜 시작하였는지 정확히 기억나지 않지만, 영화에 나오는 해커들의 모습을 동경했었던 것 같습니다.
  그래서인지 MS사의 VS 디버깅 모드에서 x86 어셈블리를 처음 접하고 독학하였고,
  대학교 수업으로 리눅스를 접하고 유닉스\&리눅스 운영체제만 6년째 사용하고 있습니다.
  그리고 이맥스가 해커들의 에디터라고 들은 이후 몇 번의 시도와 포기를 반복해 2년째 사용하고있습니다.

  % 무엇을 배웠지?
  % ~~~~~~~~~~~
  대학교 시절에는 주로 마이크로프로세서와 리눅스 커널, 유닉스\&리눅스 시스템 프로그래밍을 주로 공부하였습니다.
  기억에 남는 수업으로는 대학교 3학년 수업으로 임베디드 웹서버를 주제로 프로젝트를 진행하는 수업 이였습니다.
  리눅스 환경에서 C로 HTTP의 GET 요청을 파싱해 정적 페이지를 보여주거나 간단한 CGI를 사용하는 정도의 프로젝트였습니다.
  프로젝트를 진행하면서 기억에 남는 것들 몇 가지 있지만 가장 기억에 남는 것은 프로그래밍이 밤샘을 할정도로 재미있었다는 것 입니다. \\
  % 기억에 남는 것은 ``데드라인이 끝날 때쯤에는 새로운 기능 추가하지 말자, 백업, 모듈화 하자'' 정도.
  대학원 시절에는 OOP, TDD, 리팩토링, 디자인 패턴, 함수형 언어 등과 같이 언어 혹은 프로그래밍 자체에 관심을 가졌습니다.


  % Skills
  % ------
  \section{스킬}

  {\sl Languages \& Software:} ~C/C++, Python, Lisp, GIT \\
  {\sl Operating systems:} ~Linux.


  % Experience
  % ----------
  \section{경력사항}

  {\sl 퀄컴 카메라 지그 업무 지원} \hfill {\sl 2013.04 $\sim$ 2014.07} \vspace{1mm} \\
  {\small
    퀄컴 칩을 사용하는 삼성 핸드폰 모델에 대한 카메라 모듈센서 검증을 위한 소프트웨어 지원, \\
    주로 아래와 같은 업무를 담당함.}
  \vspace{1mm}
  \begin{itemize}
    {\small
    \item[-] 각 모델 포팅 및 유지보수. \vspace{-1mm}
    \item[-] 기능검증.}
  \end{itemize}

  {\sl SDC Micro 입사} \hfill {\sl 2012.12 $\sim$ 2014.07} \vspace{0mm}


  % Education
  % ---------
  \section{학력사항}

  {\sl 호서대 일반대학원, 컴퓨터공학 석사} \hfill {\sl 2011.02 $\sim$ 2013.02} \vspace{1mm}
  \begin{itemize}
    {\small
    \item[-] 학위논문: ~\href{http://dlibrary.hoseo.ac.kr/search/DetailView.ax?sid=4&cid=950591}
      ``스레드 생성 및 소멸 오버헤드를 줄이기 위한 히스토리기반 동적 스레드 풀'' \vspace{-1mm}
    \item[-] 학점: ~4.38/4.5}
  \end{itemize}

  {\sl 호서대학교, 컴퓨터공학과} \hfill {\sl 2004.02 $\sim$ 2011.02} \vspace{1mm}
  \begin{itemize}
    {\small
    \item[-] 학점: ~3.46/4.5}
  \end{itemize}


\end{resume}

% Document
% <-------
\end{document}
